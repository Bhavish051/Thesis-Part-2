\documentclass{article}
\usepackage[utf8]{inputenc}
\usepackage{graphicx}
\usepackage[hidelinks]{hyperref}
\usepackage{listings}
\usepackage{xcolor}

\definecolor{codegreen}{rgb}{0,0.6,0}
\definecolor{codegray}{rgb}{0.5,0.5,0.5}
\definecolor{codepurple}{rgb}{0.58,0,0.82}
\definecolor{backcolour}{rgb}{0.95,0.95,0.92}

\lstdefinestyle{codestyle}{
    backgroundcolor=\color{backcolour},   
    commentstyle=\color{codegreen},
    keywordstyle=\color{magenta},
    numberstyle=\tiny\color{codegray},
    stringstyle=\color{codepurple},
    breaklines=true,
    numbers=left,
}

\lstset{style=codestyle}

\pagestyle{headings}
\begin{document}
\begin{titlepage}
\title{TRACKING MALICIOUS TRANSACTIONS IN CRYPTOCURRENCIES}
\author{Bhavish Dhanda {\\ Supervisor: Dr Hassan Asghar} {\\ Co-Supervisor: Dr Benjamin Zhao}}
\date{}
\end{titlepage}
\maketitle
\begin{center}
    \includegraphics[width=0.7\linewidth]{logo.jpg}\\[4ex]
    Department of Computing and Engineering\\
    Macquarie University\\
    Australia
\end{center}
\pagebreak

\begin{center}
    \section*{ACKNOWLEDGEMENTS}
        \pagebreak
    \section*{STATEMENT OF CANDIDATE}
        \pagebreak
\end{center}

\tableofcontents
\pagebreak
\listoffigures
\pagebreak
\lstlistoflistings
\pagebreak

\section{Abstract}
Cryptocurrencies form what is called the new decentralized finance world of digital finances. They have become increasingly popular in the recent few years and with this technology becoming more and more accessible, the number of users of this form of payments have grown exponentially.On of the other side of coin, this mass usage of this technology gives a small percentage of people of hide in plain sight and do illegal activities. These people who are involved in such malicious activites are also supported by the anonymity features of this form of payments. 

\pagebreak

\section{Introduction}
Cryptocurrencies have become the modern form of decentralized finance giving the general public a level of visibility and control over financial systems that we could never imagine while using traditional financial systems. Their adoption has increased exponentially in the last few years with a few countries even deciding to make it a legal tender for their day to day transactions. Such countries include but are not limited to Central African Republic, El Salvador \cite{browne_2022} where Bitcoin has been deemed as a legal form of tender by the local authorities. 

The striking difference between so-called decentralized finance\cite{zetzsche_arner_buckley_2020} formed by cryptocurrencies and traditional banking systems the type of ledger they maintain, and how the currency is controlled. Decentralized as the word suggests means that the control is not the in hands of a single person or entity rather the entire network, in this case, all the people using and servicing the cryptocurrency influence all the decisions that are made which include, the price largely controlled by supply and demand and the number of people mining and other factors, validation of transactions is normally done by validation nodes inside the networks and similar functions all are done by the public nodes inside the network. The second major factor that contributes to this decentralization is the public ledger these currencies maintain in contrast to a centralized secured ledger maintained by traditional banking systems where only certain authorized users can access the ledger, in the case of cryptocurrencies, anyone can access the ledger and see all sorts of information stored in a typical currency ledger which includes account balances, transactions made etc. 

Now with this increased popularity there is a growing problem of illegal use of cryptocurrencies both through the dark web markets and also through direct P2P transactions. Sesha Kethineni1 and Ying Cao1 \cite{kethineni_cao_2019} have done a very good job to highlight the seriousness of the problem we are trying to tackle. They explain how modern cryptocurrencies such as Monero\cite{getmonero.org}, Dash\cite{dash_2022} and a few others who are built upon existing currencies with the sole purpose of improving privacy are being more and more used for illegal activities. Along with that they explain how some countries as planning to introduce cryptocurrencies as their mainstream form of payment. As much as it seems like this move this move is disliked by the more powerful countries like US, smaller countries see it as a way of achieving financial freedom US's payment systems. But this increasing accessibility gives people who are involved in illegal activities the perfect opportunity to hide in plain sight. 


\textbf{Our goal in this paper is to find ways to track down on these malicious transactions and to try and get as close as possible to the real world entity behind these activities.}

In the following sections we will go through everything mentioned above in detail starting with background about cryptocurrencies and the problem itself, followed by reviewing some work already done by researchers in this field to understand what can we learn from their work, and then we will discuss some ethical considerations made and why were they made in this piece of work. In the last few sections we will go through our methodology and the reasoning behind including some decisions made and why we made them, after which we will do some critical analysis of this work, followed by some ideas of potential future work people can undertake taking into account lessons learnt from our research and concluding with a \textbf{short and brief} conclusion about the paper.

\pagebreak
\section{Background}

\subsection{What are cryptocurrencies}
Cryptocurrencies are a modern form of digital decentralized currencies which aim to put all the control of a currency and how it works in the hands of the consumers. The distinguishing factor between the conventional currencies we normally use and crypto is first its operation of a public ledger and second decentralized control. Now, every currency needs to maintain a ledger to keep track of the consumers, their activites i.e. the transactions, their account balances, in traditional banking systems all of this information is confidential and only a select group of people have access to it whereas in the cryptocurrencies everything is public, anyone can download the blockchain on their systems or access it through a number of APIs available these days to look at this information. However with all this information, it doesn't mean that you know how much money someone has because cryptocurrencies hide their users' identity behind wallet addresses and there is no direct link between the real work identity of a person and his wallet address. So you can check how money there is in a wallet address, the transactions the address has made but cannot check who operates/owns the wallet address.

\subsection{How do cryptocurrencies work?}
\includegraphics[]{Simple Blockchain.jpeg}
Before we actually start exploring our area of research we need to be aware of how cryptocurrencies technically operate. To simplify things we will be using Bitcoin as our reference to explain the technical architecture.

As said earlier all currencies need a ledger to work so in case of cryptocurrencies it is the blockchain itself. Blockchain could be any is essentially is a Linked List \cite{} of Blocks put one after another. Each Block inside the blockchain is a primarily a collection of transaction with some other metadata associated. Modern cryptocurrencies work on maintaining a public ledger which is the distinguishing factor of these currencies with the standard paper currencies that we have been using for ages. 

In terms of actual decentralized operation of the currency, it works on the concept of a P2P network, that means in order to make a transaction on the currency you can connect directly to the other party without the oversight/involvement of a third party in the middle. Apart from that a large number of nodes/clients join the P2P network to host the currency and perform additional tasks such as mining, validation in return for some reward which depends on the currency itself. This is how the decentralization is enable the transactions are not managed or approved by a single entity or person rather a network of nodes on the P2P network known as validator nodes, same goes for miners, there are nodes on the network that perform mining and they are interacting directly with the blockchain so this eliminates the need to having to keep control of the currency in the hands of a single person or an entity.

\subsection{How do cryptocurrencies preserve privacy}

Cryptocurrencies work on the concept of wallet addresses which is similar to having a bank account at any traditional financial institutions. 

\subsection{What exactly are malicious transactions?}
In this paper we will at times refer to a transaction or address being malicious. What we mean when are saying that an address or transaction is malicious, is that the particular address or the addresses involved in a particular transaction are trying to do something illegal. These illegal activites can include money laundering, malware, scams, ponzi schemes amongst other ways to cause harm to the general public. The basic intention over here to signify that the entity controlling the address or performing a transaction is doing something which would be deemed illegal in the context of local laws and is thereby prohibited by law. 

\pagebreak

\includegraphics[width=1\linewidth]{BTC_Popularity.png}
\includegraphics[width=1.1\linewidth]{BTC_Price.png}

\pagebreak

\subsection{How serious is this problem}

As mentioned earlier this issue of illegal use of cryptocurrencies is becoming worse and worse every with a sharp increase in the last few years and also a sharp increase expected in the next couple of years at-least due to the increasing popularity of these services. Moreover this is getting worse with the introduction of privacy coins whose sole purpose is to preserve the privacy of users with the downside of a bit higher fees but this makes our point clear that a lot of people are working towards even strengthen privacy in cryptocurrencies with the main motivation being to be able to perform illegal activities on these networks with the lowest chance of getting caught. 

As we can see in the figure how the relative interest of BTC rose from its early days. There were a very few early investors in BTC which looking at the current prices yielded extremely good results. But as



\cite{kethineni_cao_2019}
\pagebreak
\section{Related Work}
In this section we will be going through the work that has been previously done in this area of research. As we will go through this section we will realise that a lot of people are already working to break through this issue of governance in cryptocurrencies. As we saw in the last section that illegal and malicious transactions in cryptocurrencies is an ever increasing problem so its very important that some action is taken to add some sort of regulation.

\subsection{De anonymization of cryptocurrencies}
Muller et al \cite{muller} tried to achieve a similar goal as ours but in a different fashion. The team identified a couple of dark web markets and went on to hunt for the vendors of illegal services. Dark web markets are extremely popular for the sale of illegal drugs, weapons and services and people involved in such activities usually tend to use these services because of the privacy they offer and with the introduction of cryptocurrencies, they have become even more anonymous. They started with public information available on the markets, which is the customer reviews, and from there tried to deduce important information such as time of transaction, value of transaction and the vendor on the market to start with and also took any other information they found. Once they were able to estimate a rough time when the transaction might have occurred they added a buffer of 1 day and looked for a transaction in the blockchain with the same order value. Now the chances that there will be another transaction with the exact value in the specified time frame are negligibly low, so it was relatively easy to mark down the transaction. After they were able to extract the transaction, they took the destination of the money and mapped it to the vendors account.

In this research they have a done a very good job in identifying the wallet addresses of the vendors, but there are a couple of things we need to critically evaluate as well. Starting with the good, a very good evaluation they did, was they took into account the payment systems of the market and changed their algorithm to identify the vendor accordingly as some markets perform direct transactions whereas some are know to use an escrow service \cite{Fig} for this. A big downside of their research is the the limited expansion. For instance, they are only working with a few dark web markets, and their research purely relies on the customer leaving reviews for the vendor so their approach would be way more effective in markets where customers are encouraged to put reviews and so are vendors to improve their visibility but in markets where this is not the case or reviews are not enforced or even the case where reviewing is not available at all, this approach would either have a very poor performance, or in the last case would completely fail. 

\subsection{Tracking money across cryptocurrency borders}
In contrast to centralized currencies like dollar, euros, cryptocurrencies, do not have a concept of borders. What that means is that unlike dollar which e.g. is the currency officially used as a legal tender in the United States, and in other countries one has to convert it into the local currency like one would have to convert their dollars into Euros in Europe if they wish to use, cryptocurrencies have no such thing. Its a decentralized form of payment used in the same way worldwide. 

Yousaf Et al. \cite{tracing_transactions_across_cryptocurrency_ledgers_-_usenix} worked to trace cryptocurrency transaction across multiple currencies. Now this is where they explored beyond the borders of cryptocurrencies.


\pagebreak

\section{Ethical Considerations}

Now that we have a good understanding of the problem at our hand and what we are trying to achieve there are a few ethical considerations we need to take into account before we actually start diving into what has been already done and what we will be doing.

The main point we need to keep in mind here is although we are working with public blockchain data most of the time, we are however trying to get to the real world entity for the associated address, and that is where these ethical concerns kick in. The reason for that is once and if we do reach the real world entity we are dealing with personal data about people which they might not always want everyone to know. This can however can explained in terms of malicious users as the authorities are chasing them up but the work we are doing can potentially be used for non-malicious addresses as well although results may vary.

\pagebreak

\section{Methodology}
 \subsection{Introduction}
    Since there is not much information about cryptocurrencies except the information in the public ledger we will need to look outside that scope and see what information we can find about addresses and transactions to make some justified conclusion so as to eventually draw a suspicion explaining how is the address malicious and the conclusions we are making to justify its link with the real world entity. 
    
    The aim over here is now to find a target address either directly or through a transaction and find all the information I can related to it. We have shortlisted a few sources which keep records of information reported by the public, so we will need to scrape through all that information and its related reports to see if we can find something which can identify the person itself. If not we will need to move to the neighbour address and perform a similar search to observe the behaviour of the neighbour address and draw some conclusions. 
    
    The next plan is to identify a few strategic address whose physical location we are aware of with complete certainty and try and find interactions with those addresses. 
    
    \subsection{Step 1 : The target address}
        The first step for us identify the target address. For our research to keep things simple we aimed at an approach which would work for all the addresses rather than specifically illegal activities. This would then enable our work to be used by other researchers in this domain who are interested to see what kind of information we can obtain through this methodology. 
        
        There are 2 ways implemented in our current research to obtain the target address: 
        \begin{enumerate}
            \item The first one, which is primarily used for our development is to pick the latest address in the latest block. 
            For this what we did was used an API 
            
            \item As an extension to the current functionality we have also given the user the option to input an address so that he can look for the information he is after rather than just picking up the latest address which is not always what one is after. 
        
        \end{enumerate}
        
        After we have obtained the address we pass it onto our next service which does the data extraction, ....
        
        
    \subsection{Step 2: Metadata about the address}
    
        \textbf{Collection of metadata about the address or transaction}. 
        This is the tricky part for us since our research purely relies on data publicly available which can be on the blockchain or information reported by private entities on the Internet. This information is very limited specially if we pick up a random address for our investigation and the reason for this is due to the pseudo-anonymous nature of cryptocurrencies. In cryptocurrencies a person can essentially set up as many addresses or so called account numbers as he wishes without any personally identifiable information in contrast to traditional banking system where a proper KYC is mandatory before one can use those services. Due to this the only information readily available is the public key of the user using the services which is his wallet address. What we can do is use that wallet address to query different sources to get as much information as possible about it. For this we be utilizing a mix of API services and also build a web scraping service to extract information no available from API due to reasons highlighted below.
        
        In case our starting point is a transaction we will be targeting one of the addresses in the transaction. The information about getting the addresses from the transaction is available readily on the internet through APIs.
        
        Our application can work both on an address or a transaction so our initial call will depend on what we are targeting. 
        
        To extract the address from the transaction we can utilize the public API available from Blockchain.info\cite{blockchain.com}. 
        
        Blockchain.info ........
        
        \textbf{Information about this source V IMP}
        
        \textbf{Url : GET} \url{https://blockchain.info/rawtx/{{tx_hash_or_id}}}
        Every transaction in the blockchain gets a hash value when generated and verified so this API takes in that as a path variable and returns the response below for a sample random transaction we picked up. The data model for the response is also shown in figure \cite{Fig}.
        
        \textbf{Sample Response: }
        \lstinputlisting[caption=A single transaction from the blockchain]{SingleTx.json}
        
        As we can see from the sample response we can directly access the input and output address as our target and move on to the next step which is getting data about the individual address itself.
        
        Now the task for us is to pick up the address in question and identify as much information as possible about it before we move on.
        
        Firstly we will be utilizing an API from Blockchain.info\cite{blockchain.com} to get basic information about the address. The API below from this source gives us
        \textbf{Url : GET} \url{https://blockchain.info/rawaddr/{{address_id}}}
        
        This API takes in the wallet address which is the publicly available on the blockchain as a path variable and gives all the information about it directly from the blockchain. The data structure of the response is shown in the figure\cite{} from which we will observe the interactions of the address inside the blockchain.
        
        \textbf{Sample Response: }
        
        \lstinputlisting[caption=A single address from the blockchain]{SingleAddr.json}
        
        The main information we need to extract from here is the interactions of the address throughout the blockchain. As we can see from the sample response we get a list of transactions in the \textbf{txs} object though which we can loop through to get to the immediate neighbours of the address. We refer immediate neighbours to addresses which directly interact with the address though a transaction where our target address can either be the sender or the receiver. 
        
\pagebreak 

    \subsection{Step 3: Web Scraping}
        Now that we have most of the metadata about the address we need to understand the nature of the address, we will starting digging into the address to find links to the real world. For this we will be building a web scraping service to go through a couple of identified online sources which seem have data which can help us get as close to the real world entity as possible. 
        
        When it comes to web scraping we will be downloading the entire page as an HTML instead of using APIs and then parsing the page element by element. This approach has both advantages and dis-advantages but in our cases the advantages outweigh the negatives so we are using this approach. Firstly when it comes to APIs, these sources are not best known to provide these services so getting an API key is very hard. We applied for an API key but its been over months since we have got anything so we decided to use the web crawling service as it will also give us more control over what data we can process. And finally APIs always have this security throttling feature against Denial of Service attacks so we can bypass this while scraping the web as we are acting as if a lot of users are accessing the web page which is a very normal behaviour and most websites do not have any restriction to the number of users who can access their page.
        
        For the web scraping of the page obtained from the sources we have to tailor write our code to adapt to the HTML returned from the page to target and extract data from the exact section of the page we are after. The main section of the page we are interested is the scam alerts \cite{}. In order to do so we are using the python's\cite{python.org} beautiful soup\cite{beautiful_soup_documentation} library which is well known to be used for these applications where we want to read exact sections from an HTML page.
        
        % Insert Code for the scraping
        
        The sources we are targeting right now are 
        \begin{enumerate}
            \item Bitcoin Who is Who \cite{bitcoinwhoswho}
            % \item Hash XP \cite{hashxp.org}
        \end{enumerate}
        In the sections below we will go though in detail the data provided by each source in detail but breifly these 2 sources provide nearly the same information except in some case one is known to have more information about the address than the other so it is always good to have 2 sources and also helps in better data validation. In the end we cannot completely rely on this information as this is publicly sourced and there are high chances of false positives as well so we have to take a cautious approach when analyzing this data and making further conclusions. 
        
        In short these sources provide some very good insights about an address such as the basic information from the blockchain such as balance(s), transaction(s), but the main data we are interested in from these sources is the scam alert which we will be scraping using our service. These are reports submitted by people from all over the world wherein an address was known to perform a malicious activity. We will analyzing that data to make further conclusions about the address.
        
        In the web scraping service what we are doing is starting to get data about the actual address itself and then moving on to the neighbours of the address identified earlier through it transactions and getting data about them to observe and understand the behaviour of the address though the neighbours and if needed might move the second neighbours as well which is the addresses to which the direct neighbours have interacted with.
        \begin{enumerate}
            \item BitcoinWhoIsWho : From this source when we download the page we get HTML of the sample page attached in fig \cite{} . From here the main data we are after is the scam alerts section which in this case a bunch of reports sent through by random people online who have either witnessed or been targeted by a particular address. 
            \item HashXp : 
        \end{enumerate}
        
        After we have extracted the data from the 2 sources we are saving it in a seperate file to analyze and understand what sort of information we can get and what conclusions we can make about the target address from the data about the address and its neighbours.

\pagebreak

    \subsection{Step 4 : Analyzing data obtained to get some results}
    
        Once we have got all the data from the sources above we dump everything into a single file to try and visualise the relationships between the target addresses and its neighbours with the vital information to make some justified reasonable conclusions. \textbf{We need to keep in mind our goal in the end is to get close to the real world entity associated with the address, not identify whether its legal or illegal as that work has already been done.} 
        
\pagebreak

    \subsection{Step 5 : Validation of Results}
        
        
\pagebreak
\section{Critical Judgement and Evaluation of Results}

\pagebreak
\section{Potential Future Work}

\pagebreak
\section{Conclusion}

\pagebreak
\bibliographystyle{ieeetr}
\bibliography{citation} 
\end{document}